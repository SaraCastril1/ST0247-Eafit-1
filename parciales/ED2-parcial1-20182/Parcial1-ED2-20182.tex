\documentclass[10 pt]{article}
\usepackage{tikz}
\usetikzlibrary{arrows}
\usepackage[margin=0.5 in]{geometry}
\usepackage[utf8]{inputenc}
\usepackage{tabu}
\usepackage{color}
\usepackage{amsmath}
\usepackage{verbatimbox}
\usepackage{xcolor}
\usepackage{listings}
\usepackage{enumitem}
\usepackage{fancyvrb}
\usepackage{multicol}
\setlength{\columnsep}{1cm} 
\title{\textbf {Estructuras de Datos 2 - ST0247\\Examen Parcial 1}}
\author{Nombre ..............................\\
		Departamento de Informática y Sistemas\\
		Universidad EAFIT\\}
\date{Septiembre 13 de 2018}
\begin{document}
\lstdefinestyle{customc}{
	language=Java, 
	numbers=left, 
	showspaces=false,
    showstringspaces=false, 
    tabsize=2, 
    breaklines=true,
    xleftmargin=5.0ex,
}
\lstset{escapechar=@,style=customc, numbers=left, stepnumber = 1} 
\maketitle
\begin{multicols}{2}
\section{Voraces 20\%}
Durante la clase de Estructuras de Datos 2, el profesor propuso el siguiente problema para que se resolviera en parejas. Te daban $n$ números $\{x_1,x_2,x_3,...,x_n\}$ y un entero $k, k\leq n$. La tarea era encontrar un subconjunto de exactamente $k$ números cuya suma fuera mínima. Sin embargo, para que el problema fuera más entretenido, el profesor prometió un bonus a la pareja que escribiera un algoritmo con una complejidad no superior a $O(n\log n)$. Para esto, las parejas 1 y 2 propusieron lo siguiente, respectivamente:
\begin{enumerate}
	\item Es imposible encontrar un algoritmo que funcione en complejidad $O(n \log n)$ o inferior  porque es necesario intentar cada subconjunto de $k$ números para determinar cual de estos tiene suma mínima.
	\item Es posible encontrar un algoritmo que funcione con complejidad $O(n \log n)$ o inferior porque, si ordenamos los elementos de una manera especifica, no es necesario intentar todos los posibles subconjuntos de $k$ elementos.
\end{enumerate} 
\begin{enumerate}[label=\alph*)]
\item (10\%) Escoja la respuesta correcta:
\begin{enumerate}[label=\roman*)]
	% Respuesta: d
	\item La pareja 1 tiene la razón, pero su justificación es falsa.
	\item La pareja 1 tiene la razón y su justificación es verdadera.
	\item La pareja 2 tiene la razón, pero su justificación es falsa.
	\item La pareja 2 tiene la razon y su justificación es verdadera.
\end{enumerate}
\item (10\%) Describa en pocas líneas el algoritmo que usted plantea para el problema anterior. Explique la complejidad de su algoritmo. 
\noindent\rule{9cm}{0.05pt}
\noindent\rule{9cm}{0.05pt}
\noindent\rule{9cm}{0.05pt}
\noindent\rule{9cm}{0.05pt}
\noindent\rule{9cm}{0.05pt}
\noindent\rule{9cm}{0.05pt}
\noindent\rule{9cm}{0.05pt}
\end{enumerate}
\section{Fuerza Bruta 20\%}

Considere el siguiente problema. Tenemos un arreglo de $n$ números no negativos $\{x_1,x_2,x_3, ..., x_n\}$. Queremos determinar si es posible encontrar un $i, 0 \leq i < n$ de tal manera que 
$\sum\limits_{j=0}^{i} x_j = \sum\limits_{j=i+1}^{n} x_j$. Ayúdanos a resolver este problema. Como un ejemplo, para $x = \{3,4,7,1,8,1,1,2,2,1\}$, la respuesta es verdadero y el $i=3$.
\begin{lstlisting}
boolean sol(int[] x){
  boolean can = false;
  int left, right;
  right=left=0;
  int n=x.length;
  for(int i=0; i<n; i++){
    left = left+x[i];
    for(int j=......; j<n; j++){
      right = right+x[j];
    }
    can = can || (.................);
    right = 0;
  }
  return can;
}
\end{lstlisting}
\begin{enumerate}[label=\alph*)]
	%Respuesta: i+1
	\item (10\%) Complete la línea 7 ...........
	%Respuesta: left == right
	\item (10\%) Complete la línea 10 ..........  
\end{enumerate}
\section{Backtracking 30\%}

El problema de la \textbf{subsecuencia común más larga} es el siguiente. Dadas dos secuencias, encontrar la longitud de la secuencia más larga presente en ambas.
Una subsecuencia es una secuencia que aparece en el mismo orden relativo, pero --no necesariamente-- de forma contigua. Como un ejemplo, ``abc'', ``abg'', ``bdf'',
``aeg'' y ``acefg'' son subsecuencias de ``abcdefg''. Entonces, para una cadena de longitud $n$, existen $2^n$ posibles subsecuencias. Este problema es utilizado
en la implementación del comando \texttt{diff}, para comparación de archivos, disponible en sistemas Unix.  También tiene muchas aplicaciones en bioinformática. \\

\noindent
Considere los siguientes ejemplos para el problema:
\begin{itemize}
\item Para ``ABCDGH'' y ``AEDFHR'' es ``ADH'' y su longitud es 3. Se retornaría 3.
\item Para ``AGGTAB'' y ``GXTXAYB'' es ``GTAB'' y su longitud es 4. Se retornaría 4.\\
\end{itemize}

Una forma de resolver este problema es usando \emph{backtracking}, como un ejemplo, para las cadenas  ``AXYT'' y  ``AYZX'', dada una función recursiva \texttt{lcs} 
que resuelve el problema, se obtendría el siguiente árbol (parcial) de recursión:

{\scriptsize
\begin{Verbatim}[commandchars=\\\{\},codes={\catcode`$=3\catcode`_=8}]
                 lcs("AXYT", "AYZX")
                 /                 \textbackslash
    lcs("AXY", "AYZX")            lcs("AXYT", "AYZ")
\_\_\_\_\_\_\_\_\_\_\_\_\_\_\_\_\_\_\_\_\_\_\_\_\_\_\_\_\_\_\_\_\_\_\_\_\_\_\_\_\_\_\_\_\_\_\_\_\_\_\_\_\_\_\_\_\_\_\_\_\_\_\_\_\_\_\_\_\_\_\_\_
    lcs("AXY", "AYZX")          
      /            \textbackslash               
lcs("AX", "AYZX") \textbf{lcs("AXY", "AYZ")}
\_\_\_\_\_\_\_\_\_\_\_\_\_\_\_\_\_\_\_\_\_\_\_\_\_\_\_\_\_\_\_\_\_\_\_\_\_\_\_\_\_\_\_\_\_\_\_\_\_\_\_\_\_\_\_\_\_\_\_\_\_\_\_\_\_\_\_\_\_\_\_\_\_
    lcs("AXYT", "AYZ")
     /             \textbackslash
\textbf{lcs("AXY", "AYZ")} lcs("AXYT", "AY")
\end{Verbatim}
}



Al siguiente código le faltan algunas lineas, complétalas por favor.

{\footnotesize
\begin{lstlisting}
int lcs(int i, int j, String s1, String s2){
 if(i == 0 || j == 0){
   return 0;   
 }
 boolean prev = i < s1.length() && j < s2.length();
 if(prev && s1.charAt(i) == s2.charAt(j)){
     return 1 + lcs(______, _______, s1, s2);
 }
 int ni = lcs(i - 1, j, s1, s2);
 int nj = lcs(i, j - 1, s1, s2);
 return Math.max(_____, _____);
}
public int lcs(String s1, String s2){
 return lcs(s1.length(), s2.length(), s1, s2);
}
\end{lstlisting}
}
\begin{enumerate}[label=\alph*)]
% i-1, j-1
 \item (10\%) Linea 7  \_\_\_\_\_\_\_\_\_\_\_\_, \_\_\_\_\_\_\_\_\_\_\_\_ \\
% ni + nj
 \item (10\%) Linea 11 \_\_\_\_\_\_\_\_\_\_\_\_, \_\_\_\_\_\_\_\_\_\_\_\_ \\
% O(2^n)
 \item (10\%) Suponga que $n$ es la suma de la longitud de las dos cadenas. El algoritmo \texttt{lcs} ejecuta, en el
 peor de los casos, $T(n)$ = \_\_\_\_\_\_\_\_\_\_\_\_\_\_\_\_\_\_\_\_ instrucciones. \\
\end{enumerate}

\textbf{Nota:} En la complejidad pueden poner la ecuación de recurrencia o la solución a la ecuación de recurrencia aplicando la notación O.

\section{Backtracking 30\%}
El problema de las $N$ reinas consiste en tomar un tablero de ajedrez de $N\times N$ y ubicar $N$ reinas de tal manera que ninguna reina quede amenazada. Una posible solución para $N=8$ sería:
\[
\begin{bmatrix}
0 & 0 & 0 & * & 0 & 0 & 0 & 0 \\
0 & 0 & 0 & 0 & 0 & 0 & * & 0 \\
0 & 0 & * & 0 & 0 & 0 & 0 & 0 \\
0 & 0 & 0 & 0 & 0 & 0 & 0 & * \\
0 & * & 0 & 0 & 0 & 0 & 0 & 0 \\
0 & 0 & 0 & 0 & * & 0 & 0 & 0 \\
* & 0 & 0 & 0 & 0 & 0 & 0 & 0 \\
0 & 0 & 0 & 0 & 0 & * & 0 & 0
\end{bmatrix}
\]
\[
solucion=\{4,7,3,8,2,5,1,6\}\ (columnas)
\]
Ayúdanos a resolver el problema anterior. La solución consiste en entregar un arreglo $a$ donde $a_i$ es un entero que indica la columna de la fila $i$ donde hay una reina.

\begin{lstlisting}
void sol(int[] a, int r) {
  int N = a.length;
  if (...........) {
    print(Arrays.toString(a));
    return;
  }
  for (int i = 0; i < N; ++i) {
    a[r]= ........;
    if(place(a,r)) sol(a, .......);
  }
}
boolean place(int[]a,int r){
  for(int i=0;i<r;++i){
    if(a[i]==a[r]) return false;
    if((a[i]-a[r]) == (r-i)) return false;
    if((a[r]-a[i]) == (r-i)) return false;
  }
  return true;
}
void print(int[]a){
  int n=a.length;
  System.out.print("[");
  for(int i=0;i<n-1;i++){
    System.out.print(a[i]+",");
  }
  System.out.println(a[n-1] +"]");
}
\end{lstlisting}
\begin{enumerate}[label=\alph*)]
	% Respuesta: r == N
	\item (10\%) Complete la línea 3 ..............
	% Respuesta: a[r] = i
	\item (10\%) Complete la línea 8 ..............
	% Respuesta: r + 1
	\item (10\%) Complete la línea 9 ..............
\end{enumerate}
\end{multicols}

\end{document}